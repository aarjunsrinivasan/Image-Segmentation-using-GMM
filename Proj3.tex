\documentclass{article}
\usepackage{graphicx}
\usepackage[center]{caption}
% If you're new to LaTeX, here's some short tutorials:
% https://www.overleaf.com/learn/latex/Learn_LaTeX_in_30_minutes
% https://en.wikibooks.org/wiki/LaTeX/Basics

% Formatting
\usepackage[utf8]{inputenc}
\usepackage[margin=1in]{geometry}
\usepackage[titletoc,title]{appendix}

% Math
% https://www.overleaf.com/learn/latex/Mathematical_expressions
% https://en.wikibooks.org/wiki/LaTeX/Mathematics
\usepackage{amsmath,amsfonts,amssymb,mathtools}
\usepackage{bm}
\usepackage{siunitx}

% Images
% https://www.overleaf.com/learn/latex/Inserting_Images
% https://en.wikibooks.org/wiki/LaTeX/Floats,_Figures_and_Captions
\usepackage{graphicx,float}

% Tables
% https://www.overleaf.com/learn/latex/Tables
% https://en.wikibooks.org/wiki/LaTeX/Tables

% Algorithms
% https://www.overleaf.com/learn/latex/algorithms
% https://en.wikibooks.org/wiki/LaTeX/Algorithms
\usepackage[ruled,vlined]{algorithm2e}
\usepackage{algorithmic}

% Code syntax highlighting
% https://www.overleaf.com/learn/latex/Code_Highlighting_with_minted
\usepackage{minted}
\usemintedstyle{borland}

% References
% https://www.overleaf.com/learn/latex/Bibliography_management_in_LaTeX
% https://en.wikibooks.org/wiki/LaTeX/Bibliography_Management
\usepackage{biblatex}
\addbibresource{references.bib}

\DeclareMathOperator{\taninv}{tan\,inverse}

% Title content
\title{ENPM 673 Project 2}
\author{Arjun Srinivasan Ambalam, Praveen Menaka Sekar, Arun Kumar Dhandayuthabani}
\begin{document}

\maketitle

% Introduction and Overview
\section{Introduction}
The aim of this project is to apply color segmentation using Gaussian Mixture Models(GMM) and
Expectation Maximization(EM) techniques. Since the buoys have been captured
underwater, change in lighting intensities and noise make it difficult to employ the use of
conventional segmentation techniques involving color thresholding. Hence, this is
accomplished with different methods explained and provided in the pipeline of the project
guidelines. Below you will see explanations of what we did to accomplish such an arduous
task.


\section{Data Preparation}


\section{Plotting Average Histograms}
\begin{enumerate}
    \item Average histogram is drawn by averaging all the histograms for each frame i.e. first the histogram
is drawn for each frame and then all data from all the histograms are averaged together to make an
average histogram for the whole video.
    \item Average histogram is shown for each of the channels i.e. RGB for each buoy.
    \item We notice the number of pixels along the peak in each channel from the observed average histograms.
    \item For orange buoy the number of pixels along the peak of the red channel were higher, around 80
more than the blue and green channel, hence only red channel is used for the orange buoy for further
processing.
    \item Similarly for green buoy the peak along green channel was higher than the peak about blue and red
channel, hence only green channel for the green buoy is used for further processing.
    \item For the yellow buoy, the peak about the red and green channels were comparable, hence the average of
the red and green channel is used for the further processing.
\end{enumerate}

\section{1D Gaussian}
\begin{enumerate}
    \item After calculating the average histogram we now aim to fit a 1D Gaussian over it.
    \item We first load the train data set and look at the channel of which the histogram is of maximum intensity
for that buoy.
    \item That is for the orange buoy we see that the average histogram has the maximum intensity in red
channel. For yellow it is both red and green and for green it is only in green channel.
    \item After loading that particular channel we calculate the mean and standard deviation of all the images
of train set for that buoy.
    \item This mean and standard deviation is then used to generate the Gaussian using the equation -
    
\end{enumerate}



\section{Buoy Detection using N 1D Gaussian}

\begin{enumerate}
    \item  We have identified that one dimensional Gaussian will give better results and
since we can fit it to the histogram of that particular colour will give us a better understanding of the
segmentation in terms of noise and actual buoy.
    \item For the orange buoy, we have taken the histogram and plotted the three 1-D Gaussian and we notice
that it completely matches the average color histogram of red channel.
    \item Similarly we compute the same process for the rest of the buoys to and we keep of generating the
Gaussian until it clearly fits the histogram.
    \item The images of the histogram along with the Gaussian used to fit has been depicted below.
    \item Once the Gaussian have been generated to fit the histogram the Y axis represent the probabilities.
    \item We segment every pixels according to the (probabilities) thus stating a threshold on the probability of
that pixel being in that Gaussian.
    \item We create a binary image after applying threshold according to the probabilities and find the contour
of that buoy.
    \item After this point we trace back the contour on the original frame of the video, and it is depicted in the
images below.
\end{enumerate}

\section{Gaussian Mixture Model and Expectation Maximization}

\begin{enumerate}
    \item  A Gaussian mixture model is a probabilistic model that assumes all the data points are generated from
a mixture of a finite number of Gaussian distributions with unknown parameters. It can be thought
that the GMM is a generalized k-means cluster to incorporate information about the variance structure
of the data and the centers of the Gaussian formed.
    \item The main difficulty in GMM algorithm is that the data is unlabeled, so one cannot determine which
point came from which latent component. Expectation-maximization algorithm solves this problem
through iterative process. At first we select random components (i.e., randomly selected center points
and a common standard deviations) and compute for each point the probability of being generated
by each component of the model. Then we tweak the parameters i.e., mean and standard deviations
of each Gaussian formed for each considered channels. Repeating this process guarantees to always
converge to a local optimum values

\end{enumerate}

\section{Discussions}
\begin{enumerate}
    \item We used three 1-D Gaussian on the red channel of the orange buoy as they gave good results.
    \item Using three 1-D Gaussian on the green channel of the green buoy gave good results with some
background error. Background error can be removed by using morphological operations such as dilation
and erosion. Background error can also be removed by using contouring operations. But the downside of
only using contouring operation is that it does not work on all the frames. Using contouring operations
after morphological operations gave us good results.
    \item Yellow buoy was the most difficult to identify as the yellow color background error due to sunlight is
present in the initial frames of the video.
    \item Only using three 1-D Gaussian on on one of the channels of the yellow buoy gave no results, hence we
used three 1-D Gaussian on the average of the green and red channel. Using contouring operations
after morphological operations (erosion and dilation) gave us good results, with only some background
error in the initial frames.
    \item Using different color space such as HSV and RGBY can be used for this project. RGBY has its
own yellow channel, hence will make it easier to detect yellow color buoy in the input video.
    \item We also tried to convert the image from RGB to LAB and performed CLAHE(Contrast Limited Adaptive Histogram Equalization) which provided similar results to that of Average Histogram.

\end{enumerate}

\end{document}